\section{Usage}
\subsection{Files and directories}

Explain:
\begin{itemize}
\item Soucre code directories and files
\item Model directories and files
\end{itemize}

\subsection{Basic commands}

Explain:
\begin{itemize}
\item ./createmodel
\item ./configure
\item make
\item Mathematica interface (see models/<model>/start.m)
\end{itemize}

\subsubsection{\code{createmodel}}

The \code{createmodel} script sets up a \flexisusy\ model.  This involves
%
\begin{itemize}
\item a model directory \code{models/<flexiblesusy-model>/}
\item a makefile module \code{models/<flexiblesusy-model>/module.mk}
\item a \flexisusy\ model file \code{models/<flexiblesusy-model>/FlexibleSUSY.m}
\item a Mathematica start script \code{models/<flexiblesusy-model>/start.m}
\end{itemize}
%
Usage:
\begin{lstlisting}[language=bash]
./createmodel --models=<flexiblesusy-model>:<sarah-model>
\end{lstlisting}
Here \code{<flexiblesusy-model>} is the name of \flexisusy\ model
to be created and \code{<sarah-model>} is the name of the \sarah\
model file which defines the Lagrangian and the particles.\\
Example:
\begin{lstlisting}[language=bash]
./createmodel --models=MyCMSSM:MSSM
\end{lstlisting}
%
Multiple models can be created at once by separating the models by a
comma.\\
Example:
\begin{lstlisting}[language=bash]
./createmodel --models=MyCMSSM:MSSM,MyCNMSSM:NMSSM
\end{lstlisting}
%
If a certain \sarah\ sub-model should be used, the name of the sub-model
has to be appended with at preceeding slash.\\
Example: use the \code{CKM} sub-model of the \code{MSSM}
\begin{lstlisting}[language=bash]
./createmodel --models=MyCMSSM:MSSM/CKM
\end{lstlisting}
%
For further information and options see
\begin{lstlisting}[language=bash]
./createmodel --help
\end{lstlisting}

\subsubsection{\code{configure}}

The \code{configure} script checks for the installed compilers,
libraries and Mathematica.  If all of these exists in a sufficent
version, the \code{Makefile} is created, which contains the information
on how to compile the code.  The user has to specify the models which
should be included in the build via the \code{--with-models=} option,
e.g.
%
\begin{lstlisting}[language=bash]
./configure --with-models=<flexibsusy-model>
\end{lstlisting}
%
Here \code{<flexibsusy-model>} is either \code{all} or a comma
separated list of \flexisusy models.

Furthermore, the user can select which RG solver algorithms to use via
the \code{--with-algorithms=} option.\\
Example:
%
\begin{lstlisting}[language=bash]
./configure --with-algorithms=two_scale
\end{lstlisting}

The \code{configure} script further allows to select the C++ and
Fortran compilers, the Mathematica kernel command as well as paths to
libraries and header files.  See
\code{configure --help} for all available options.\\
Example:
%
\begin{lstlisting}[language=bash]
./configure --with-models=MSSM --with-cxx=clang++
   --with-boost-incdir=/usr/include/
   --with-boost-libdir=/usr/lib/
\end{lstlisting}

\subsection{Model files}

Explain how to write a \flexisusy\ model file (starting from a \sarah\
model file).

A \flexisusy\ model file is a Mathematica file, where the following
pieces are defined:

\paragraph{General model information}

\subparagraph{FSModelName}
Name of the \flexisusy\ model.
Example (MSSM):
\begin{lstlisting}[language=Mathematica]
FSModelName = "MSSM";
\end{lstlisting}

\subparagraph{OnlyLowEnergyFlexibleSUSY} If set to True, creates a
spectrum generator without high-scale constraint, i.e.\ only a
low-scale and susy-scale constraint will be created (default: False).
In this case all model parameters, except for gauge and Yukawa
couplings are input at the susy scale.  This option is similar to
\code{OnlyLowEnergySPheno} in SARAH/SPheno.
\\
Example:
\begin{lstlisting}[language=Mathematica]
OnlyLowEnergyFlexibleSUSY = False;  (* default *)
\end{lstlisting}

\paragraph{Input and output parameters}

\subparagraph{MINPAR} This list of two-component lists defines model
input parameters and their SLHA keys.  These parameters will be read
from the MINPAR block in the SLHA file.
\\
Example (MSSM):
\begin{lstlisting}[language=Mathematica]
MINPAR = { {1, m0},
           {2, m12},
           {3, TanBeta},
           {4, Sign[\[Mu]]},
           {5, Azero}
         };
\end{lstlisting}

\subparagraph{EXPAR} This list of two-component lists defines further
model input parameters and their SLHA keys.  These parameters will be
read from the EXTPAR block in the SLHA file.
\\
Example (E6SSM):
\begin{lstlisting}[language=Mathematica]
EXTPAR = { {61, LambdaInput},
           {62, KappaInput},
           {63, muPrimeInput},
           {64, BmuPrimeInput},
           {65, vSInput}
         };
\end{lstlisting}

\subparagraph{DefaultParameterPoint} This list of two-component lists
defines default values for the model input parameters.
\\
Example (MSSM):
\begin{lstlisting}[language=Mathematica]
DefaultParameterPoint = {
    {m0, 125},
    {m12, 500},
    {TanBeta, 10},
    {Sign[\[Mu]], 1},
    {Azero, 0}
};
\end{lstlisting}

\paragraph{Constraints}
Currently three constraints are supported: low-scale, susy-scale and
high-scale constraints.  In \flexisusy\ they are named as
\code{LowScale}, \code{SUSYScale} and \code{HighScale}.  For
each constraint there is a scale definition (named after the
constraint), an initial guess for the scale (concatenation of the
constraint name and \code{FirstGuess}) and a list settings to be
applied at the constraint (concatenation of the constraint name and
\code{Input}).
\\
Example (MSSM):
\begin{lstlisting}[language=Mathematica]
(* susy-scale constraint *)
SUSYScale = Sqrt[M[Su[1]]*M[Su[6]]];        (* scale definition *)
SUSYScaleFirstGuess = Sqrt[m0^2 + 4 m12^2]; (* first scale guess *)
SUSYScaleInput = {};                        (* nothing is set here *)

(* high-scale constraint *)
HighScale = g1 == g2;                       (* scale definition *)
HighScaleFirstGuess = 1.0 10^14;            (* first scale guess *)
HighScaleInput = {
   {T[Ye], Azero*Ye},
   {T[Yd], Azero*Yd},
   {T[Yu], Azero*Yu},
   {mHd2, m0^2},
   {mHu2, m0^2},
   {mq2, UNITMATRIX[3] m0^2},
   {ml2, UNITMATRIX[3] m0^2},
   {md2, UNITMATRIX[3] m0^2},
   {mu2, UNITMATRIX[3] m0^2},
   {me2, UNITMATRIX[3] m0^2},
   {MassB, m12},
   {MassWB,m12},
   {MassG, m12}
};

LowScale = SM[MZ];                          (* scale definition *)
LowScaleFirstGuess = SM[MZ];                (* first scale guess *)
LowScaleInput = {
   {vd, 2 MZDRbar / Sqrt[GUTNormalization[g1]^2 g1^2 + g2^2] Cos[ArcTan[TanBeta]]},
   {vu, 2 MZDRbar / Sqrt[GUTNormalization[g1]^2 g1^2 + g2^2] Sin[ArcTan[TanBeta]]}
};
\end{lstlisting}

\paragraph{Initial parameter guesses} At the \code{LowScale} and
\code{HighScale} it is recommended to make an initial guess for the
model parameters.  This can be done via
\begin{lstlisting}[language=Mathematica]
InitialGuessAtLowScale = {
   {vd, SM[vev] Cos[ArcTan[TanBeta]]},
   {vu, SM[vev] Sin[ArcTan[TanBeta]]}
};

InitialGuessAtHighScale = {
   {\[Mu]   , 1.0},
   {B[\[Mu]], 0.0}
};
\end{lstlisting}

\paragraph{Setting the tree-level EWSB eqs.\ solution by hand} In case
the meta code cannot find an analytic solution to the tree-level EWSB
eqs., one can set solutions by hand in the model file using the
variable \code{TreeLevelEWSBSolution}.
\\
Example (MSSM):
\begin{lstlisting}[language=Mathematica]
TreeLevelEWSBSolution = {
  {\[Mu]   , ... },
  {B[\[Mu]], ... }
};
\end{lstlisting}

\begin{sidewaystable}[tb]
  \centering
  \begin{tabularx}{\textwidth}{>{\ttfamily}l>{\ttfamily}lX}
    \toprule
    variable    & default value & description \\
    \midrule
    FSModelName & Model`Name & Name of the \flexisusy\ model \\
    OnlyLowEnergyFlexibleSUSY & False & low-energy model \\
    MINPAR & \{\} & list of input parameters in SLHA MINPAR block \\
    EXTPAR & \{\} & list of input parameters in SLHA EXTPAR block \\
    DefaultParameterPoint & \{\} & list of default values for the input parameters \\
    LowScale & SM[MZ] & Standard Model matching scale (in GeV) \\
    LowScaleFirstGuess & SM[MZ] & first guess for the Standard Model matching scale (in GeV) \\
    LowScaleInput & \{\} & settings applied at the low scale \\
    SUSYScale & 1000 & scale of supersymmetric particle masses (in GeV) \\
    SUSYScaleFirstGuess & 1000 & first guess for the susy scale (in GeV) \\
    SUSYScaleInput & \{\} & settings applied at the susy scale \\
    HighScale & SARAH`hyperchargeCoupling == SARAH`leftCoupling
      & unification scale (in GeV) \\
    HighScaleFirstGuess & $10^{14}$ & first guess for the unification scale (in GeV) \\
    HighScaleInput & \{\} & settings applied at the unification scale \\
    \bottomrule
  \end{tabularx}
  \caption{\flexisusy\ model file variables}
  \label{tab:model-file-variables}
\end{sidewaystable}

\subsection{Makefile modules}

Explain how to write a custom makefile module (module.mk) for a
\flexisusy\ model.
