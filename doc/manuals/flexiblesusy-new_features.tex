\documentclass[final,3p,11pt,pdflatex]{elsarticle}
\usepackage[utf8x]{inputenc}      % input font encoding
\usepackage{amsmath,amssymb,mathtools}
\usepackage[T1]{fontenc}          % output font encoding
\usepackage{booktabs,tabularx}
\usepackage{graphicx,subfig}
\usepackage{xspace}
\usepackage[usenames]{xcolor}\definecolor{fscolor}{RGB}{44,118,255}
\usepackage{tikz,tikz-uml}
\usetikzlibrary{arrows,calc,matrix,positioning,shapes}
\usepackage{listings}
\usepackage[absolute]{textpos}
\usepackage[many]{tcolorbox}
\usepackage{xparse}
\usepackage[font=small,labelfont=bf,format=plain,margin=0.05\textwidth]{caption}
\usepackage{bbm}
\bibstyle{elsarticle-num}
% source code highlighting
\lstset{breaklines=true,
  breakatwhitespace=true,
  stepnumber=1,
  basicstyle=\ttfamily\footnotesize,
  commentstyle=\ttfamily\color{gray},
  prebreak={\textbackslash},
  breakindent=10pt,
  breakautoindent=false,
  showspaces=false,
  showstringspaces=false,
  frame=shadowbox,
  rulesepcolor=\color{gray},
  rulesep=0.1em,
  abovecaptionskip=0em,
  aboveskip=1.5em,
  belowcaptionskip=0.5em,
  belowskip=1em,
}

\geometry{top=2.2cm,left=3cm,right=3cm,bottom=4cm,footskip=3em}

\usepackage[pdftitle={New features in FlexibleSUSY since version 2.0},
pdfauthor={Peter Athron, Markus Bach, Dylan Harries, Wojciech
  Kotlarski, Thomas Kwasnitza, Jae-hyeon Park, Dominik Stockinger,
  Alexander Voigt, Jobst Ziebell},
pdfkeywords={FlexibleSUSY,supersymmetry,spectrum,generator,MSSM,NMSSM,E6SSM},
bookmarks=true, linktocpage, colorlinks=true, allbordercolors=white,
allcolors=fscolor]{hyperref} \biboptions{sort&compress}
\allowdisplaybreaks

%macros
\newcommand{\modelname}[1]{\texttt{#1}\@\xspace}
\newcommand{\sarah}{\texttt{SARAH}\@\xspace}
\newcommand{\spheno}{\texttt{SPheno}\@\xspace}
\newcommand{\suspect}{\texttt{SuSpect}\@\xspace}
\newcommand{\isasusy}{\texttt{ISASUSY}\@\xspace}
\newcommand{\suseflav}{\texttt{SuSeFLAV}\@\xspace}
\newcommand{\nmspec}{\texttt{NMSPEC}\@\xspace}
\newcommand{\fs}{\texttt{FlexibleSUSY}\@\xspace}
\newcommand{\fsbreak}{\texttt{Flex\-ib\-le\-SUSY}\@\xspace}
\newcommand{\HSSUSY}{\modelname{HSSUSY}}
\newcommand{\susyhd}{\texttt{SusyHD}\@\xspace}
\newcommand{\MhEFT}{\texttt{MhEFT}\@\xspace}
\newcommand{\softsusy}{\texttt{SOFTSUSY}\@\xspace}
\newcommand{\micromegas}{\texttt{micrOMEGAS}\@\xspace}
\newcommand{\darksusy}{\texttt{DarkSUSY}\@\xspace}
\newcommand{\calchep}{\texttt{CALCHEP}\@\xspace}
\newcommand{\lanhep}{\texttt{LANHEP}\@\xspace}
\newcommand{\madgraph}{\texttt{MadGraph}\@\xspace}
\newcommand{\whizard}{\texttt{WHIZARD}\@\xspace}
\newcommand{\sherpa}{\texttt{SHERPA}\@\xspace}
\newcommand{\TSIL}{\texttt{TSIL}\@\xspace}
\newcommand{\LoopTools}{\texttt{LoopTools}\@\xspace}
\newcommand{\FAFC}{\texttt{FeynArts}\@\xspace}
\newcommand{\helac}{\texttt{HELAC}\@\xspace}
\newcommand{\herwig}{\texttt{HERWIG}\@\xspace}
\newcommand{\pythia}{\texttt{PYTHIA}\@\xspace}
\newcommand{\checkmate}{\texttt{CheckMate}\@\xspace}
\newcommand{\madanalysis}{\texttt{MadAnalysis}\@\xspace}
\newcommand{\smodels}{\texttt{SModelS}\@\xspace}
\newcommand{\fastlim}{\texttt{Fastlim}\@\xspace}
\newcommand{\colliderbit}{\texttt{ColliderBit}\@\xspace}
\newcommand{\higgsbounds}{\texttt{HiggsBounds}\@\xspace}
\newcommand{\higgssignals}{\texttt{HiggsSignals}\@\xspace}
\newcommand{\htm}{\texttt{H3m}\@\xspace}
\newcommand{\Himalaya}{\texttt{Himalaya}\@\xspace}
\newcommand{\FeynHiggs}{\texttt{FeynHiggs}\@\xspace}
\newcommand{\FeynRules}{\texttt{FeynRules}\@\xspace}
\newcommand{\NMSSMCalc}{\texttt{NMSSMCalc}\@\xspace}
\newcommand{\multinest}{\texttt{MultiNest}\@\xspace}
\newcommand{\CPsuperH}{\texttt{CPsuperH}\@\xspace}
\newcommand{\GAMBIT}{\texttt{GAMBIT}\@\xspace}
\newcommand{\GMTCalc}{\texttt{GM2Calc}\@\xspace}
\newcommand{\fsone}{\fs 1.0\@\xspace}
\newcommand{\fstwo}{\fs 2.0\@\xspace}
\newcommand{\fsh}{\texttt{FlexibleSUSY+}\Himalaya\xspace}
\newcommand{\fbsm}{\texttt{FlexibleBSM}\@\xspace}
\newcommand{\fcpv}{\texttt{FlexibleCPV}\@\xspace}
\newcommand{\fmw}{\texttt{FlexibleMW}\@\xspace}
\newcommand{\famu}{\texttt{FlexibleAMU}\@\xspace}
\newcommand{\fsas}{\texttt{FlexibleSAS}\@\xspace}
\newcommand{\feft}{\texttt{Flex\-ib\-le\-EFT\-Higgs}\@\xspace}
\newcommand{\mathematica}{\texttt{Ma\-the\-ma\-ti\-ca}\xspace}
\newcommand{\LibraryLink}{\texttt{LibraryLink}\xspace}
\newcommand{\ESSM}{E$_6$SSM\@\xspace}
\newcommand{\code}[1]{\lstinline|#1|}  % inline source code
\newcommand{\ol}[1]{\overline{#1}}
\newcommand{\MSbar}{\ensuremath{\ol{\text{MS}}}\xspace}
\newcommand{\DRbar}{\ensuremath{\ol{\text{DR}}}\xspace}
\newcommand{\MDRbar}{\ensuremath{\ol{\text{MDR}}}\xspace}
\newcommand{\unit}[1]{\,\text{#1}}      % units
\newcommand{\userinput}{\text{input}}
\newcommand{\pole}{\text{pole}}
\newcommand{\Lagr}{\mathcal{L}}
\newcommand{\SM}{\ensuremath{\text{SM}}\xspace}
\newcommand{\BSM}{\ensuremath{\text{BSM}}\xspace}
\newcommand{\THDM}{\ensuremath{\text{THDM}}\xspace}
\newcommand{\THDMII}{\ensuremath{\text{THDM-II}}\xspace}
\newcommand{\MSSM}{\ensuremath{\text{MSSM}}\xspace}
\newcommand{\MS}{\ensuremath{M_S}\xspace}
\newcommand{\MLCP}{\ensuremath{M_\text{LCP}}\xspace}
\newcommand{\MEWSB}{\ensuremath{M_\text{EWSB}}\xspace}
\newcommand{\Qmatch}{\ensuremath{Q_\text{match}}}
\newcommand{\QpoleBSM}{\ensuremath{Q_\text{pole,BSM}}}
\newcommand{\QpoleSM}{\ensuremath{Q_\text{pole,SM}}}
\newcommand{\QED}{\ensuremath{\text{QED}}}
\newcommand{\unity}{\mathbf{1}}
\newcommand{\amu}{\ensuremath{a_\mu}\xspace}
\newcommand{\amuBSM}{\ensuremath{\amu^{\BSM}}\xspace}
\newcommand{\amuMSSM}{\ensuremath{\amu^{\MSSM}}\xspace}
\newcommand{\edm}[1]{\ensuremath{d_{#1}}\xspace}
\newcommand{\edmBSM}[1]{\ensuremath{\edm{#1}^{\BSM}}\xspace}
\newcommand{\deltaVB}{\ensuremath{\delta_{\text{VB}}}\xspace}
\newcommand{\figref}[1]{\figurename~\ref{#1}}
\newcommand{\secref}[1]{Section~\ref{#1}}
\newcommand{\appref}[1]{Appendix~\ref{#1}}
\newcommand{\tabref}[1]{\tablename~\ref{#1}}
\newcommand{\exref}[1]{Example~\ref{#1}}
\newcommand{\ptitle}[1]{\emph{#1}}
\renewcommand{\ptitle}[1]{}
\newcommand{\Zv}{\mathbf{\backslash}\mkern-11.0mu{Z}}
\newcommand{\scoeff}[2]{[ c^{#1}_{#2}(Q) ]}
\newcommand{\barlog}{\overline{\log}}
\newcommand{\GSM}{\ensuremath{G_{\SM}}\xspace}
\newcommand{\CP}{\ensuremath{CP}\xspace}
\newcommand{\azero}{\ensuremath{A_0}\xspace}
\newcommand{\mhalf}{\ensuremath{M_{1/2}}\xspace}
\newcommand{\mzero}{\ensuremath{m_0}\xspace}
\newcommand{\SQCD}{SUSY-QCD\xspace}

\newcommand{\multilinecell}[2][c]{%
  \begin{tabular}[#1]{@{}c@{}}#2\end{tabular}}

\newtcolorbox[
  auto counter
]{example}[1][]{%
  breakable,
  before skip=1em,
  after skip=1em,
  enhanced,
  colback=white,
  colbacktitle=white,
  arc=0pt,
  leftrule=1.5pt,
  rightrule=0.5pt,
  toprule=0.5pt,
  bottomrule=0.5pt,
  titlerule=0.5pt,
  colframe=fscolor,
  fonttitle=\bfseries\normalcolor,
  overlay={},
  parbox=false,
  title=Example~\thetcbcounter,
  #1
}

% comments
\newcommand{\avnote}[1]{\textcolor{blue}{AV: #1}}
\newcommand{\dhnote}[1]{\textcolor{green}{DH: #1}}
\newcommand{\panote}[1]{\textcolor{red}{PA: #1}}
\newcommand{\jpnote}[1]{\textcolor{magenta}{JP: #1}}
\newcommand{\mbnote}[1]{\textcolor{orange}{MB: #1}}

\NewDocumentEnvironment{OptionTable}{m m O{llXX}}{%
\table[tbh!]
  \tabularx{\textwidth}{#3}%
    \toprule
    Symbol & Default value & Allowed values & Description \\
    \midrule
}{\endtabularx\caption{#1}\label{#2}\endtable}

\makeatletter
\lstnewenvironment{numlstlisting}[1][]{%
  \lstset{%
    #1,
    numbers=left,
    firstnumber=auto,
    numberstyle=\tiny\sffamily}%
  \csname\@lst @SetFirstNumber\endcsname
}{%
  \csname \@lst @SaveFirstNumber\endcsname
}
\makeatother
\DeclareMathOperator{\diag}{diag}
\DeclareMathOperator{\sign}{sign}
\DeclareMathOperator{\re}{Re}
\DeclareMathOperator{\im}{Im}
\def\at{\alpha_t}
\def\ab{\alpha_b}
\def\as{\alpha_s}
\def\atau{\alpha_{\tau}}
\def\aem{\alpha_{\text{em}}}
\def\oat{O(\at)}
\def\oab{O(\ab)}
\def\oatau{O(\atau)}
\def\oatab{O(\at\ab)}
\def\oatas{O(\at\as)}
\def\oabas{O(\ab\as)}
\def\oatababq{O(\at\ab + \ab^2)}
\def\oatqatababq{O(\at^2 + \at\ab + \ab^2)}
\def\oatasatq{O(\at\as + \at^2)}
\def\oatasabas{O(\at\as +\ab\as)}
\def\oatasabasatq{O(\at\as + \at^2 +\ab\as)}
\def\oatq{O(\at^2)}
\def\oabq{O(\ab^2)}
\def\oatauq{O(\atau^2)}
\def\oabatau{O(\ab \atau)}
\def\oatplusabsq{O((\at+\ab)^2)}
\def\oas{O(\as)}
\def\oatauqatab{O(\atau^2 +\ab \atau )}

\journal{Computer Physics Communications}
\begin{document}
\begin{frontmatter}
 \vspace*{0.5cm}
 \title{\Large\bf New features in FlexibleSUSY since version 2.0}

\author[Monash]{Peter Athron}
\author[dresden]{Markus Bach}
\author[adelaide,prague]{Dylan Harries}
\author[dresden]{Wojciech Kotlarski}
\author[dresden]{Thomas Kwasnitza}
\author[kiasquc]{Jae-hyeon Park}
\author[dresden]{Dominik St\"ockinger}
\author[rwth]{Alexander Voigt\corref{cor1}}
\ead{Alexander.Voigt@physik.rwth-aachen.de}
\cortext[cor1]{Corresponding author}
\author[dresden]{Jobst Ziebell}
\address[Monash]{ARC Centre of Excellence for Particle Physics at
  the Terascale, School of Physics, Monash University, Melbourne,
  Victoria 3800, Australia}
\address[dresden]{Institut f\"ur Kern- und Teilchenphysik,
TU Dresden, Zellescher Weg 19, 01069 Dresden, Germany}
\address[adelaide]{ARC Centre of Excellence for Particle Physics at
the Terascale, Department of Physics, The University of Adelaide,
Adelaide, South Australia 5005, Australia}
\address[prague]{Institute of Particle and Nuclear Physics, Faculty of
  Mathematics and Physics, Charles University in Prague, V
  Hole\v{s}ovi\v{c}k\'{a}ch 2, 180 00 Praha 8, Czech Republic}
\address[kiasquc]{Quantum Universe Center,
Korea Institute for Advanced Study,
85 Hoegiro Dongdaemungu,
Seoul 02455, Republic of Korea}
\address[rwth]{Institute for Theoretical Particle Physics and Cosmology, RWTH Aachen University, 52074 Aachen, Germany}

  \begin{abstract}
    We document major new features and improvements of \fs
    \cite{Athron:2014yba,Athron:2017fvs}, a \mathematica and C++
    package with a dependency on the external package \sarah, that
    generates fast and precise spectrum generators.
  \end{abstract}

\begin{keyword}
sparticle,
supersymmetry,
Higgs,
renormalization group equations
\PACS 12.60.Cn, 12.60.Fr, 12.60.Jv, 12.60.-i
\PACS 14.80.Bn, 14.80.Da, 14.80.Ec, 14.80.Fd, 14.80.Ly, 14.80.Nb, 14.80.Sv
\end{keyword}
\end{frontmatter}

% report numbers
\begin{textblock*}{10em}(\textwidth,1.5cm)
\raggedleft\noindent\footnotesize
KIAS--Q17043 \\
TTK--17--31 \\
CoEPP--MN--17--16
\end{textblock*}

\clearpage
\newgeometry{top=2.5cm,left=3cm,right=3cm,bottom=4cm,footskip=3em}
\section*{New version program summary}
\noindent
{\em Program title:} \fs\\[0.5em]
{\em Licensing provisions:} GPLv3\\[0.5em]
{\em Program obtainable from:} \url{http://flexiblesusy.hepforge.org/}\\[0.5em]
{\em Programming language:} C++, Wolfram/Mathematica, FORTRAN, Bourne shell\\[0.5em]
{\em Operating system:} Tested on FreeBSD, Linux, Mac OS X, Windows/Cygwin\\[0.5em]
{\em External routines:} \sarah, Boost library, Eigen library, GNU Scientific Library\\[0.5em]
{\em Optional external routines:} BLAS, Himalaya, LAPACK, LoopTools, SQLite, TSIL\\[0.5em]
{\em Journal reference of previous version:} Comput.Phys.Commun.\ 230 (2018) 145-217\\[0.5em]
{\em Does the new version supersede the previous version?:} yes \\[0.5em]
{\em Reasons for the new version:} Program extension including new models, observables and algorithms\\[0.5em]
{\em Summary of revisions:} $\mu\to e\gamma$\\[0.5em]
{\em Nature of problem:}
%
Determining the mass spectrum, mixings and further observables for an
arbitrary extension of the Standard Model, input by the user. The
generated code must find simultaneous solutions to constraints that
are specified at two or more different renormalization scales, which
are connected by renormalization group equations forming a large set
of coupled first-order differential
equations. \\[0.5em]
%
{\em Solution method:}
%
Nested iterative algorithm and numerical
minimization of the Higgs potential.\\[0.5em]
%
{\em Restrictions:}
%
The couplings must remain perturbative at all scales between the
highest and lowest boundary condition.  Tensor-like Lagrangian
parameters of rank 3 are currently not supported.  The automatic
determination of the Standard Model-like gauge and Yukawa couplings is
only supported for models that have the Standard Model gauge group
$SU(3)_C\times SU(2)_L\times U(1)_Y$ as a gauge symmetry group factor.
However, due to the modular nature of the generated code, adapting and
extending it to overcome restrictions in scope is quite straightforward.

\clearpage
\tableofcontents

\clearpage
\section{External libraries}

\subsection{New mandatory external libraries}

Since version X.Y.Z \fs requires the following additional libraries to
be installed:
%
\begin{itemize}
\item \LoopTools: required for calculating the 1-loop decays.
\end{itemize}

\subsection{New optional external libraries}

\begin{itemize}
\item \TSIL: Since version X.Y.Z \fs can calculate 2-loop corretions
  to SM \MSbar top quark mass of the order $O(\at\as + \at^2)$ from
  Ref.~\cite{Martin:2016xsp}.  The calculation of these corrections
  requires linking the \TSIL library \cite{Martin:2005qm} to \fs.
\end{itemize}

\clearpage
\section{$\mu\to e\gamma$}

\section*{References}
\addcontentsline{toc}{section}{References}

\bibliographystyle{JHEP}
\bibliography{flexiblesusy-new_features}

\end{document}
